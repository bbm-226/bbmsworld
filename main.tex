\documentclass{article}
\usepackage[UTF8]{ctex}
\usepackage{geometry}
\usepackage{multirow}
\usepackage{natbib}
\geometry{left=3.18cm,right=3.18cm,top=2.54cm,bottom=2.54cm}
\usepackage{graphicx}
\pagestyle{plain}	
\usepackage{setspace}
\usepackage{enumerate}
\usepackage{caption2}
\usepackage{datetime} %日期
\renewcommand{\today}{\number\year 年 \number\month 月 \number\day 日}
\renewcommand{\captionlabelfont}{\small}
\renewcommand{\captionfont}{\small}
\begin{document}

\begin{figure}
    \centering
    \includegraphics[width=8cm]{upc.png}

    \label{figupc}
\end{figure}

	\begin{center}
		\quad \\
		\quad \\
		\heiti \fontsize{45}{17} \quad \quad \quad 
		\vskip 1.5cm
		\heiti \zihao{2} 《计算科学导论》个人职业规划
	\end{center}
	\vskip 2.0cm
		
	\begin{quotation}
% 	\begin{center}
		\doublespacing
		
        \zihao{4}\par\setlength\parindent{7em}
		\quad 

		学生姓名:\underline{\qquad  陈泊蓂 \qquad}
	
	学\hspace{0.61cm} 号:\underline{\qquad 2007010207\quad}
	
	专业班级:\underline{   本研一体人工智能类}
		
        学\hspace{0.61cm} 院:\underline{计算机科学与技术学院}
% 	\end{center}
		\vskip 1.5cm
		\centering
		\begin{table}[h]
            \centering 
            \zihao{4}
            \begin{tabular}{|c|c|c|c|c|c|c|c|c|}
            % 这里的rl 与表格对应可以看到,姓名是r,右对齐的;学号是l,左对齐的;若想居中,使用c关键字。
                \hline
                \multicolumn{5}{|c|}{分项评价} &\multicolumn{2}{c|}{整体评价}  & 总    分 & 评 阅 教 师\\
                \hline
                自我 & 环境 & 职业 & 实施 & 评估与 & 完整性 & 可行性 &\multirow{2}*{} &\multirow{2}*{}\\
                分析& 分析& 定位 & 方案 & 调整 & 20\% & 20\% & ~&~ \\\            
                10\% & 10\% & 15\% & 15\% & 10\% & &  &~ &~\\
                \cline{1-7} 
                & & & & & & & ~&~ \\
                & & & & & & & ~&~ \\
                \hline      
            \end{tabular}
        \end{table}
		\vskip 2cm
		\today
	\end{quotation}

\thispagestyle{empty}
\newpage
\setcounter{page}{1}
% 在这之前是封面,在这之后是正文
\section{自我分析}
\subsection{自然条件}
我性别男,现在年龄为19岁,身高174cm,体重63kg,两眼近视度数为350°左右,身体健康,没有影响工作生活的特殊疾病,家乡位于广西钦州市并居住于此,现在在中国石油大学(华东)学习。
\subsection{性格分析}
本人具有以下性格特点:\par
1.是一个较为开朗乐观的人,在遭遇挫折后虽然可能会陷入到低谷期之中,但很快就能够从悲伤之中脱离,并以此为动力与警示来督促自己来更好的努力与学习。\par
2.是一个认真而严谨且负有责任感的人,在分配到任务后会积极的完成自己的任务,并追求能够将其高质量的完成。若是任务没有达到预期的效果,也会想方设法来提高其质量。\par
3.现在还是不太喜欢在公众面前展示自我的优点与特长,在展现自我方面还没有着积极的意识。\par
4.在学习与做任务时习惯与独自完成,没有能够与同伴很好的交流与分享方法与经验。\par
5.缺乏创造性思维,在遇到开放性问题时不能很好的提出一针见血的观点和很可行的解决方法。
\subsection{教育与学习经历}
本人所受教育经历都是在国内,还没有出国学习交流的经验,高中,初中,小学以及幼儿园都是在家乡广西钦州来学习。所获得的教育资源相对于大城市的同学来说是比较少的,在见识与眼光方面可能与周围同学相比较来说还是比较浅显。在初中之前报过一些兴趣班,如:绘画,书法等,但都未深入学习,只有书法简单的学了几年。现在在位于山东青岛的中国石油大学(华东)学习。
\subsection{工作与社会阅历}
本人至今还未在社会上有过实习等方面的经验目前还未有着工作方面的经验。也由于来自与边远地区,所获得的教育资源以及各方面较少,在社会阅历方面还是比较欠缺的,在大学以及今后的学习中也是要多关注社会上的事实,多与学长学姐多交流,来填补自己的这方面知识的空缺。
\subsection{知识、技能与经验}
由于本人在进入大学之前并未有系统全面的来学习与计算机有关的各个方面的知识,现在在计算机的专业知识方面还没有较好的了解与认识,在专业知识与技能上还需要在接下的学习中不断努力。经验方面的话还要利用大学的机会,积极参与到各种科研与学术交流中,锻炼自己,不断积累经验提前积累好经验,以便能够提前适应将来社会生活与工作中的各种情景。
\subsection{兴趣爱好与特长}
本人的兴趣爱好较为广泛,有关与体育方面的都可以参与到其中,我现在所有着经验的体育项目就有很多,如:篮球,足球,乒乓球,羽毛球等。我也对电子游戏方面比较感兴趣,主要是我个人在竞技类的游戏方面也算是比较有天赋的,比如现在较为流行的英雄联盟,王者荣耀都能够很好的掌握。特长主要就是篮球方面,我是从小学一二年级就开始打篮球了,在小学,初中的时候都是校篮球队的一员,但在高中的时候打球比较少。我个人在篮球方面技术还勉强能看过去,但身体素质相比于其他打球的人来说没有那么强壮。
\section{环境分析}
\subsection{社会环境分析}
当今社会正处于百年未有之大变局,世界格局已经由西方主导逐步转变为东西方平衡。一百多年来,中国取得了快速的进步,我们们认真学习了西方的优秀经验和制度安排。改革开放以后,中国学习先进经验,发展市场经济,逐步具备了强大的市场竞争力,中国逐步开始崛起了。中国特色社会主义道路打破了西方模式一统天下的局面。\par
中国的发展速度日益加快,美国近年来不断打压中国的发展。从贸易战到限制对华为出口芯片,中国在良好的反战前景下也面临了新的挑战。在当今的全球化环境下,世界市场化的规模和速度大大提高。虽然在今年来遭受新冠疫情的打击,世界经济发展有着倒退的趋势,但依然阻止不了当今世界经济的快速发展。在这种多交流的世界经济形势下,计算机人才市场需求潜力仍然很大 计算机专业人才的市场需求具有很大的潜力。第一:高端应用型人才的需求潜力比较大。在当前产业结构升级的推动下,高端应用型人才的需求量正在不断增加,随着云计算、大数据、物联网、人工智能等技术平台的落地应用,高端应用型人才的需求潜力还是非常大的。第二:新技术领域的人才需求潜力比较大。目前大型科技公司纷纷开始在云计算、大数据、人工智能等领域布局,而这些新的技术领域会逐渐构建起新的技术生态,所以行业领域会需要大量掌握这些新技术的专业人才。第三:全栈型人才需求潜力比较大。工业互联网时代是平台化时代,平台化时代要求技术人员能够基于技术平台来完成行业创新,此时行业领域会需要技术人员具有更全面的技术能力,全栈化人才会更受欢迎。
\subsection{家庭环境分析}
本人目前大学在读,未婚,经济目前靠家人供给,还未实现经济的独立。家人之前希望能够回到家乡安逸的工作生活,但本人想要到外面的城市中闯荡,寻求自己的机遇。家族传统并未有什么特别的方面。
\subsection{职业环境分析}
我目前想要发展的职业方向是软件开发工程师。软件开发这一方面,从我国软件产业市场发展的进程来看,目前还只能算是起步阶段,与发达国家相比,我们之间的距离还很大。因此无论是我们的软件市场还是软件企业都需要予以精心的培育。事实上,中国软件产业正在以超常规的发展速度在世界上占有一席之地。\par
软件开发工程师的工作内容可以简要概括如下:1、指导程序员的工作。
2、参与软件工程系统的设计、开发、测试等过程。
3、协助工程管理人保证项目的质量。
4、负责工程中主要功能的代码实现。
5、解决工程中的关键问题和技术难题。
6、协调各个程序员的工作,并能与其它软件工程师协作工作。\par
软件开发工程师的工作不仅仅是要满足用户需求,更应该要在开发中知道自己要做出有怎么样效果的软件。你首先要知道做这个项目是为了解决什么问题;测试案例中应该输入什么数据等等,为了清楚地知道这些需求,要经常的与经理以及顾客交流,了解顾客的具体需求,再加以实现。\par
软件开发工程师在IT行业中越来越受到重视,其薪资也节节高升.综合数据表明,软件工程师是近期企业缺口最大的职位,可以说软件工程师是目前IT行业求职者的最佳选择。

\subsection{地域与人际环境分析}
我的工作城市如果可以选择,将从青岛,深圳,上海等地中选择,青岛这个城市的气候四季分明,夏无酷暑,冬少严寒;降水适中,热量充足;春夏多雾,冬春风大。青岛有着海洋文化和啤酒文化。青岛的海洋文化历史悠久,源远流长,资源丰富,独富特色。青岛的发展前景可以说是较为乐观的,虽然说在网络上有着一些人批评青岛的执政理念和执政能力,但青岛有着环境,有着人口优势,交通缺陷已被大大弥补,自然环境比较好,比较适宜居住,土地资源也没有那么紧张。我现在青岛的中石大读书,在毕业如果在这里工作,可以依靠在大学里认识的同学来发展自己的人脉关系,在青岛这个城市中更好的寻求自己的机遇。
\par 
\section{职业定位}
\subsection{行业领域定位与理由}
软件开发工程师分成三类:前端开发工程师、后端开发工程师、全栈开发工程师。软件开发工程师就业涉及行业:小程序开发、网站开发、软件工具开发、互联网企业管理型工具、应用软件后台开发、网页游戏开发、交易系统开发等。其中的软件工具开发和网页游戏开发比较我比较感兴趣。人在从事自己感兴趣的工作时能够更好的进行工作与保持积极向上的心态来完成好所接到的每一个项目。软件开发工程师还能经常接触到新鲜的事物,能够更好的让我来获得新鲜感,也经常接触到其他的同样志同道合的有想法的朋友。在公司中获得一定的话语权后还能够考虑转型成管理人员,比如项目经理,比如产品经理等等,来依靠自己在工作之中积累的技术和经验来摆脱写枯燥的打代码的生活。软件工程师对于我来说可以说得上是比较符合我这个人的职业了。
\par
\subsection{职业岗位起点定位与理由}
在大学的学习生活中,我要充分利用好课余的空闲时间来学习更多的知识,学习成为软件开发工程师所需要的如Java,DHP,数据库等等专业知识。如果有机会进入到公司之中实习,我的主要目标是学习和掌握公司产品或者项目的基本技术、工具和流程。在掌握了基本的技术技能和经验后,选择好自己专业或管理的发展方向,与自己的同事建立良好的关系,再向高层次的综合管理和企业战略方向发展。
\subsection{职业目标与可行性分析}
\par
\begin{enumerate}[(1)]
	\item 短期目标(大学4年)\par
	掌握好专业知识,如:计算机硬件原理,操作系统原理,编译原理,数据结构和算法分析等,并提高自身的代码水平,熟练掌握好C语言,Java。并在这四年里锻炼自己的交往能力,与学校里不止本专业的同学建立起更大的人际交往圈子。
	\item 中长期目标(5-10年)。\par
	找机会进入到公司之中进行实习,在实习期间向公司中的前辈请教,学习并查漏补缺,如果发现自己在哪些知识上还没有了解应尽快的花时间补上。我的目标是能够在公司之中一步一步的干起,并逐渐提高自己的地位,能够成为一个高层管理人员,我也希望我的年薪先能达到20w一年,后面再继续努力,去追求更高的职位和薪水。
\end{enumerate}

\section{实施方案}
\begin{enumerate}[1、]
	\item 现在我在中石大的本研一体化人工智能类学习,有着其他同学没有的学业导师和学校资源的优势。在学校学习的这段时间里要有效的利用好现有的资源和优势来不断提高自己的专业知识水平,并在自己感兴趣的领域,利用好课余时间来提前学习好知识。
	\item 在大学的学习时间里,要多与身边的同学交流知识以及学习方法,遇到问题多向老师请教,并锻炼自己的交流水平,不再像以前一样在多人的场合下不敢发表自己的见解。还要敢于在集体之中展现自己的能力,没有能力就不能让别人信服你跟随你。
	\item 在大学这个圈子里我要多与本校以及其他学校中的同学建立起关系,来扩展自己的交往圈子,不再只局限于过去的熟人,要跳出自己的舒适圈,去到还未曾涉及的领域去开拓自己的视野。
	\item 在今后的工作上即使非常忙碌,也好处理好家庭与工作之间的关系,在平时里给到家人们应有的爱和关心。如果不是在家乡工作,要注意与父母多交流,关心父母的身体以及生活状况。
	\item 在遇到工作上的压力时,不能总是独自抱怨生闷气,如果可以的话,要与家人们交流并讨论好如何处理工作上的各种问题与压力。在平时里也要多注意锻炼身体,我自己就是一个比较喜欢运动的人,如果生活或工作上有着压力,都可以通过运动来放松以缓解压力,实现身心健康。
\end{enumerate}
\par
\section{评估与调整}
\subsection{评估时间}
现制定的评估时间为每学年一次,每两年评估一次。
\subsection{评估内容}

从自己这段时间内学习到了什么,参与了哪些项目,获得了什么奖励奖项,自己这时候与上一次评估时的进步是否可见,有没有担任某一个社团的职位,在其中有没有着统率能力与个人声望。若是有着没有能够达到的目标,需要反思自己是否在这段时间内好好努力学习知识,并要做出反思与实际行动。
\subsection{调整原则}
若是自身没有能够为了目标而付诸行动与努力,则需以此为戒来提醒自己。但若是制定的计划对于自己来说并不是切实可行的,则需要对制定好的计划进行调整与改变。再次制定计划时要考虑到与当下的形势来说是否有利,且更需要考虑计划的实施行,不能立下过于缥缈不可及的计划,这样只会打击自己的积极性。




\end{document}
